%
% Carátula oficial de 75.02 Algoritmos y Programación I, cátedra Cardozo.
%
% Basado en el template realizado por Diego Essaya, disponible en
%                                                         http://lug.fi.uba.ar
% Modificado por Michel Peterson.
% Modificado por Sebastián Santisi.

%
% Acá se define el tamaño de letra principal:
%
\documentclass[12pt]{article}

%
% Título y autor(es):
%
\title{Trabajo Práctico N\b o X}
\author{Apellido1\\Apellido2}

%------------------------- Carga de paquetes ---------------------------
%
% Si no necesitás algún paquete, comentalo.
%

%
% Definición del tamaño de página y los márgenes:
%
\usepackage[margin=1in]{geometry}

%
% Vamos a escribir en castellano:
%
\usepackage[spanish]{babel}
\usepackage[utf8]{inputenc}


\usepackage{array}
\usepackage{tabularx}
\usepackage{longtable}
\usepackage{ltxtable}

\usepackage{changepage}

%
% Si preferís el tipo de letra Helvetica (Arial), descomentá las siguientes
% dos lineas (las fórmulas seguirán estando en Times):
%
%\usepackage{helvet}
%\renewcommand\familydefault{\sfdefault}

%
% El paquete amsmath agrega algunas funcionalidades extra a las fórmulas. 
% Además defino la numeración de las tablas y figuras al estilo "Figura 2.3", 
% en lugar de "Figura 7". (Por lo tanto, aunque no uses fórmulas, si querés
% este tipo de numeración dejá el paquete amsmath descomentado).
%
\usepackage{amsmath}
\numberwithin{equation}{section}
\numberwithin{figure}{section}
\numberwithin{table}{section}

%
% Para tener cabecera y pie de página con un estilo personalizado:
%
\usepackage{fancyhdr}

%
% Para poner el texto "Figura X" en negrita:
% (Si no tenés el paquete 'caption2', probá con 'caption').
%
\usepackage[hang,bf]{caption}

%
% Para poder usar subfiguras: (al estilo Figura 2.3(b) )
%
%\usepackage{subfigure}

%
% Para poder agregar notas al pie en tablas:
%
%\usepackage{threeparttable}

%------------------------------ graphicx ----------------------------------
%
% Para incluir imágenes, el siguiente código carga el paquete graphicx 
% según se esté generando un archivo dvi o un pdf (con pdflatex). 

% Para generar dvi, descomentá la linea siguiente:
%\usepackage[dvips]{graphicx}

% Para generar pdf, descomentá las dos lineas seguientes:
\usepackage[pdftex]{graphicx}
\pdfcompresslevel=9

%
% Todas las imágenes están en el directorio tp-img:
%
\newcommand{\imgdir}{includes}
\graphicspath{{\imgdir/}}
%
%------------------------------ graphicx ----------------------------------

% Necesitas este paquete si haces los diagrámas de flujo en el prográma Dia 
%\usepackage{tikz}


%------------------------- Inicio del documento ---------------------------

\begin{document}
	
	%
	% Hago que en la cabecera de página se muestre a la derecha la sección,
	% y en el pie, en número de página a la derecha:
	%
	\pagestyle{fancy}
	\renewcommand{\sectionmark}[1]{\markboth{}{\thesection\ \ #1}}
	\lhead{}
	\chead{}
	\rhead{\rightmark}
	\lfoot{}
	\cfoot{}
	\rfoot{\thepage}
	
	%
	% Carátula:
	%
	\begin{titlepage}
		
		\thispagestyle{empty}
		
		\begin{center}
			\includegraphics[scale=0.3]{fiuba}\\
			\large{\textsc{Universidad de Buenos Aires}}\\
			\large{\textsc{Facultad De Ingeniería}}\\
			\small{Año 2021 - 2\textsuperscript{do} Cuatrimestre}
		\end{center}
		
		\vfill
		
		\begin{center}
			\Large{\underline{\textsc{Organización de Datos (75.06)}}}
		\end{center}
		
		\vfill
		
		\begin{tabbing}
			\hspace{2cm}\=\+TRABAJO PRÁCTICO Nº2\\
			TEMA: \\
			FECHA DE ENTREGA: 8/12/2021\\% \today\\
			\\
			INTEGRANTES:\hspace{-1cm}\=\+\hspace{1cm}\=\hspace{6cm}\=\\
			DE LUCA ANDREA, Felipe	\>\>- \#88888\\
			FOPPIANO, Elián	\>\>- \#88888\\
		\end{tabbing}
		
		\vfill
		
		\hrule
		\vspace{0.2cm}
		
		\noindent\small{75.06 - Organización de Datos}
		
	\end{titlepage}
	
	%
	% Hago que las páginas se comiencen a contar a partir de aquí:
	%
	\setcounter{page}{1}
	
	%
	% Pongo el índice en una página aparte:
	%
	\tableofcontents
	\newpage
	
	%
	% Inicio del TP:
	%
	\newgeometry{margin=0.2in}
	
	\section{Preprocesamientos}
	
	\begingroup
	\fontsize{11pt}{12pt}\selectfont
	
	\LTXtable{\textwidth}{plantilla_preprocesamientos.tex}
	
	\endgroup

	\pagebreak

	\section{Modelos}

	\begingroup
	\fontsize{11pt}{12pt}\selectfont
	
	\LTXtable{\textwidth}{plantilla_metricas.tex}
	
	\endgroup
	
	
	\restoregeometry
	
	
	
	\pagebreak
	
\end{document}

