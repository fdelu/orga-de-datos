\newcolumntype{Y}{>{\small\raggedright\arraybackslash}X}
\renewcommand{\arraystretch}{1.5}
\noindent
\begin{longtable}{|>{\setlength\hsize{0.25\hsize}}X|>{\setlength\hsize{0.45\hsize}}X|>{\setlength\hsize{0.3\hsize}}X|}
\hline
Nombre del procesamiento & Descripción & Nombre de la función \\
\hline
Común & 
Drop de \textit{llovieron\_hamburguesas\_hoy} \newline
Reemplazar \textit{si} y \textit{no} por 1 y 0 en variable target \newline
Drop de muestras con missings en variable target \newline
Drop de muestras con datos inválidos \newline
Corrección de typos en nombres de features \newline
Si el parámetro \textbf{fecha\_to\_int} es True, convierte \textit{dia} en un entero con formato AAAAMMDD
&
\textbf{common()} \\
\hline
Día a mes &
Convierte \textit{dia} en el mes correspondiente a la fecha &
\textbf{dia\_a\_mes()} \\
\hline
Viento trigonométrico &
Convierte las features de dirección del viento en pares de features $\langle\sin{\theta},\cos{\theta}\rangle$, donde $\theta$ es el ángulo de dicha dirección &
\textbf{viento\_trigonometrico()} \\
\hline
Barrios a comunas &
Realiza un hashing de los valores de la feature \textbf{barrio}, donde el nombre de cada barrio se reemplaza por el nombre de su comuna (\textit{Comuna 1}, \textit{Comuna 2}, etc) &
\textbf{barrios\_a\_comunas()} \\
\hline
Estandarización &
Agrega un \textbf{StandardScaler()} a la \textbf{Pipeline}, que transforma los datos tal que su distribución tenga esperanza 0 y varianza 1 (deben ser todas las features numéricas) &
\textbf{standarizer()} \\
\hline
Imputador simple &
Agrega un \textbf{SimpleImputer()} a la \textbf{Pipeline}, que imputa los valores faltantes con el promedio &
\textbf{simple\_imputer()} \\
\hline
Imputador iterativo &
Agrega un \textbf{IterativeImputer()} a la \textbf{Pipeline}, que imputa los valores faltantes a partir de regresiones iterativas &
\textbf{iterative\_imputer()} \\
\hline
Drop categóricas &
Dropea las features categóricas del dataset (Si se ejecutó el preprocesamiento \textit{Viento trigonometrico}, solo dropea la feature \textbf{barrio}) &
\textbf{drop\_categoricas()} \\
\hline
Hashing trick &
Transforma una feature categórica en varias columnas con valores 0 o 1 a partir de un vector obtenido con una función de hash sobre la variable &
\textbf{hashing\_trick()} \\
\hline
Drop poco importantes &
Dado una lista de scores y un threshold, dropea las features cuyos scores (importancia de feature) sea menor al threshold &
\textbf{drop\_poco\_importantes()} \\
\hline
Feature selection &
Dado un modelo y un porcentaje de features a mantener, selecciona las features que maximicen el score del modelo dado mediante forward selection &
\textbf{feature\_selection()} \\
\hline
\caption{Preprocesamientos} \\
\end{longtable}
